%*******************************************************
% Abstract
%*******************************************************
%\renewcommand{\abstractname}{Abstract}
\pdfbookmark[1]{Abstract}{Abstract}
\begingroup
\let\clearpage\relax
\let\cleardoublepage\relax
\let\cleardoublepage\relax

\chapter*{Abstract}
The goal of this document is divided into several pieces. Mainly I want to publish and share by experience developing programme with the purpose to communicate with the \textbf{Dreamcheerky I.O.C Rocket Launcher} via USB. There was a lot of source code on the internet, but unfortunately nothing worked for me. So I started to do my own development.It is written in C and designed for Linux. Especially tested under Linux Mint 17. 

But there are further outlooks for the project and I also want to document them (e.g. face recognition, alix embedded port, movement with an arduino robot, \dots). Moreover, this document is a small evaluation of Tex and the classicthesis package provided by Andre Miete in order to use it as template for upcoming academic documents.

\vfill

\pdfbookmark[1]{Zusammenfassung}{Zusammenfassung}
\chapter*{Zusammenfassung}
Das Ziel dieses Dokuments ist in mehrere Abschnitte gegliedert. Haupts�chlich m�chte ich meine Erfahrungen bei der Entwicklung eines Programms teilen, das mit einem \textbf{Dreamcheerky I.O.C Rocket Launcher} �ber USB kommuniziert. Im Internet gibt es zwar viel Quelltext. Aber Nichts das in meinem Fall funktionierte. So fing ich an selbst zu entwickeln. Es ist in C geschrieben und f�r Linux entworfen. Speziell unter Linux Mint 17 getestet.

Aber es gibt weitere Ausblicke f�r das Projekt und ich m�che sie ebenfalls dokumentieren. (z.B. Gesichtserkennung, Alix embedded Portierung, Bewegung mit einem Arduino Roboter, \dots). Zudem ist dieses Dokument eine kleine Evaluation von Tex und Andre Miede`s classicthesis Pakets mit dem Hintergedanken es als Vorlage f�r kommende akademische Dokumente zu verwenden.

\endgroup			

\vfill