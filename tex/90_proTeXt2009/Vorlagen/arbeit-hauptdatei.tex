%% Vorlage einer l�ngeren wissenschaftlichen Arbeit f�r das Buch
%% "Wissenschaftliche Arbeiten schreiben mit LaTeX"
%% von Joachim Schlosser, 2006-2009
%%
%% Sie d�rfen dieses Dokument frei verteilen, ver�ndern und verwenden.
%%
%% Dies ist nur eine Vorlage. 
%% Zwei Kommentarzeichen (%%) bedeuten, dass eine Erl�uterung folgt.
%% Ein Kommentarzeichen heisst, dass die Zeile deaktiviert wurde
%% Die meisten Pakete sind deaktiviert, Sie aktivieren Sie, indem Sie
%% das Kommentarzeichen am Beginn der Zeile entfernen.
%% Aktivieren Sie nur, was Sie wirklich brauchen!
%% Lesen Sie in l2tabu nach, ob nicht einige Pakete mittlerweile
%% veraltet sind, wenn Sie diese Vorlage erst 2010 oder sp�ter verwenden!

\RequirePackage{fix-cm}        %% Korrektur der Computer Modern Schriftfamilien
\documentclass[%
%draft,                        %% um sp�ter den Zeilenumbruch zu kontrollieren
a4paper,                       %% DIN A4-Papier
BCOR4mm,                       %% Bindekorrektur 4 mm
DIVcalc,                       %% Satzspiegel berechnen
11pt,                          %% Schriftgr��e
%bibtotoc,                      %% Literaturverzeichnis ins Inhaltsverzeichnis
%idxtotoc,                      %% Index ins Inhaltsverzeichnis
tablecaptionabove,             %% Tabellen�ber"= statt unterschriften
]{scrreprt}                    %% KOMA-Skript Report als Klasse
\usepackage{geometry}          %% damit auch das Papierformat passt
\usepackage[english,german,ngerman]{babel}  %% Deutsch als Hauptsprache
\input{dehyphtex.tex}          %% zus�tzliche deutsche Trennmuster
%\usepackage{microtype}         %% Optischer Randausgleich mit pdfLaTeX
\usepackage[T1]{fontenc}       %% Schriftkodierung, die nativ Umlaute unterst�tzt
\usepackage[latin1]{inputenc}  %% Umlaute direkt eingeben
% \usepackage[osf]{mathpazo}    %% Palatino als Brotschrift, mit Minuskelziffern
% \linespread{1.08}             %% Palatino braucht gr��eren Zeilenabstand
% \usepackage[scaled=.87]{luximono} %% anderer Monotype-Font
\usepackage{textcomp}          %% besondere Textzeichen
\usepackage{amssymb}           %% Mathesymbole
\usepackage{amsmath}           %% erweiterte Mathemakros
%\usepackage{paralist}          %% Aufz�hlungen im Flie�text
%\usepackage{mdwlist}           %% engere Aufz�hlungen und Nummerierungen
%\usepackage{filecontents}      %% eingebundene Dateien (f�r ltxtable)
%\usepackage{ltxtable}          %% lange Tabellen mit flexibler Spaltenbreite
%\usepackage{tabularx}          %% Tabellen mit flexibler Spaltenbreite 
\usepackage{booktabs}          %% sch�nere Tabellenlinien
\usepackage{graphicx}          %% Graphiken einbinden
%\usepackage{tikz}              %% TikZ/pgf zum Schreiben von Grafiken
% \usepackage{ifpdf}            %% direktes Einbinden von Metapost-Graphiken
% \ifpdf 
% \DeclareGraphicsRule{*}{mps}{*}{} 
% \fi

%% Literaturverzeichnisse: Hier ist Babelbib gew�hlt. W�hlen Sie, was
%% Sie brauchen.
\usepackage{babelbib}          %% Mehrsprachiges Literaturverzeichnis
%\usepackage[ngerman,vario]{fancyref} %% "plain" f�r einfachere Querverweise, ",vario" f�r Seitenzahl
%% Anpassung f�r Querverweiserkennung: Kapitel durch "cha:" anstatt "chap:"
%\fancyrefchangeprefix{\fancyrefchaplabelprefix}{cha}
%\vrefwarning                   %% Varioref vertrauen, erst am Ende!

%\usepackage{index}             %% Index vorbereiten
%\makeindex                     %% make index

\usepackage[%
  bookmarks,                   %% PDF-Lesezeichen
  bookmarksopen=true,          %% Lesezeichenbaum aufgeklappt...
  bookmarksopenlevel=1,        %% ...um eine Ebene
  bookmarksnumbered=true,      %% Lesezeichen numerieren
%  pagebackref,                 %% Links von Literatur in Text
  pdfusetitle,                 %% LaTeX-Titelei als Metainfo nehmen
  pdfstartpage={1},            %% mit welcher Seite das PDF �ffnen
  pdfstartview={FitH},         %% Zoom auf Seitenbreite
  pdfkeywords={},              %% Stichw�rter f�rs PDF, kommagetrennt
  pdfsubject={},               %% Themenbeschreibung kurz
  pdfcreator={LaTeX with KOMA-Script and hyperref package},
%   pdfpagelabels,               %% Seitenzahlen wie im Dokument...
%   plainpages=false,            %% ...und nicht wie sonst
  hyperfootnotes=true,         %% Links auf Fu�noten
  hyperindex=true,             %% Indexeintr�ge verweisen auf Text
  linkbordercolor={0 1 1},     %% Rahmenfarbe interne Links
  menubordercolor={0 1 1},     %% Rahmenfarbe Literaturlinks
  urlbordercolor={1 0 0}       %% Rahmenfarbe externe Links
]{hyperref}                    %% Hyperlinks und Lesezeichen in PDF 
%\usepackage[figure]{hypcap}    %% Springt bei Links auf Abb. an die richtige Stelle

\pagestyle{headings}           %% lebende Kolumnentitel

\hyphenation{%
%  eventuelle Trennmusterdefinitionen, mit - als Trennstellenmarker  
}

%\setcounter{secnumdepth}{3}    %% Nummerierung bis zur Ebene subsubsection
%\setcounter{tocdepth}{3}       %% Inhaltsverzeichnis bis zur Ebene subsubsection 
%% Anpassungen f�r das Layout: Float-Pages
%\renewcommand{\floatpagefraction}{.7}  % vorher: .5

%% Gegen die meisten Overfull hboxes, aus
%% dctt, Axel Reichert, Message-ID: <a84us0$plqcm$7@ID-30533.news.dfncis.de>
\tolerance 1414
\hbadness 1414
\emergencystretch 1.5em
\hfuzz 0.3pt
\widowpenalty=10000
\vfuzz \hfuzz
\raggedbottom

%% Hier beginnt das eigentliche Dokument
\begin{document}

%% Die Eckdaten des Dokuments, wegen Babel erst nach \begin{document}
\author{Max Muster}
\title{Meine wissenschaftliche Arbeit}
\subject{Masterarbeit}
\publishers{Technische Universit�t Musterstadt\\
  Fakult�t f�r angewandte Wissenschaft}
\date{24.\,12.~2009}

%% Titel daraus erzeugen
\maketitle

%% Zusammenfassung. Oft auch erst nach dem Inhaltsverzeichnis.
% ohne Nummerierung
\chapter*{Zusammenfassung}
\label{cha:zusammenfassung}

Dies ist die Zusammenfassung der Arbeit. Sie sollte nicht mehr als
ein bis maximal zwei Seiten umfassen.

%%% Local Variables: 
%%% mode: latex
%%% TeX-master: "arbeit-hauptdatei"
%%% End: 


%% Inhaltsverzeichnis
\tableofcontents 

%% Abbildungsverzeichnis
%\listoffigures

%% Tabellenverzeichnis
%\listoftables

\chapter{Einf�hrung}
\label{cha:einfuhrung}

Das ist ein wenig einf�hrender Text, der --~ohne n�her darauf
einzugehen~-- auch mal einen \emph{Einschub} enthalten darf. Auch ohne
Auf\/lagen.

\section{�bersicht}
\label{sec:ubersicht}

Eine �bersicht �ber die Arbeit.

Wir haben mehreres:
\begin{itemize}
\item Theorien,
\item Beispiele,
\item Anwendungen und
\item vieles mehr.
\end{itemize}

\section{Problemstellung}
\label{sec:problemstellung}

Eine Beschreibung der Problemstellung.

%%% Local Variables: 
%%% mode: latex
%%% TeX-master: "arbeit-hauptdatei"
%%% End: 

\chapter{Theorie}
\label{cha:theorie}

Die Arbeit basiert im wesentlichen auf der Theorie des begrenzten
Wissens \cite{schlosser06:latexbuch}. Die Grundlagen des begrenzten
Wissens sowie diese Theorie wollen wir im Folgenden er�rtern.

\section{Grundlagen}
\label{sec:grundlagen}

Prinzipiell gilt, dass
\begin{equation}
  x=y+z
\end{equation}
unter der Annahme $x$ und $y$ als Zahlenma� von Textgr��e, $z$ als
Repr�sentation der Aufnahmef�higkeit.

\begin{equation*}
  \label{eq:pythagoras}
  \left.\begin{aligned}
  c^2&=a^2+b^2\\
  a^2&=p\cdot c\;\wedge\;
     b^2=q\cdot c\\
  h^2&=p\cdot q
  \end{aligned}\right\}
\begin{gathered}
  \text{Satzgruppe}\\
  \text{des Pythagoras}
\end{gathered}
\end{equation*}

\section{Theorie der B�ume}
\label{sec:theorie-der-baume}

Schon \cite{knuth:texbook} schreibt:

\begin{table}[htbp]
  \centering
  \caption{Vier Zahlen}
  \label{tab:vierzahlen}
  \begin{tabular}{lr}
    \toprule
    Zahl & Nummer \\
    \midrule
    Eins & Zwei \\
    Drei & Vier \\
    \bottomrule
  \end{tabular}
\end{table}

Mehr dazu finden Sie in Abschnitt~\ref{sec:seiteneffekte}.

\section{Erweiterungen der Theorie}
\label{sec:erwe-der-theor}

\begin{figure}[htbp]
  \centering
  %% Hier brauchen Sie eine existierende Grafikdatei
%  \includegraphics[width=.7\linewidth]{pinguin}
  \caption{Der Pinguin Tux}
  \label{fig:pinguin}
\end{figure}

\begin{figure}[htbp]
  \centering
  %% Eine Grafik in der Syntax von TikZ. Paket aktivieren!
%   \begin{tikzpicture}
%     \node {root}
%     child {node {left}}
%     child {node {right}
%       child {node {child}}
%       child {node {child}}
%     };
%   \end{tikzpicture}
  \caption{Ein Baum}
  \label{fig:einbaum}
\end{figure}

Das k�nnen wir machen durch
\begin{enumerate}
	\item etwas,
	\item etwas anders oder
	\item ganz etwas anderes.
\end{enumerate}

\begin{description}
	\item[Nichts] ist alles.
	\item[Alles] ist nichts.
\end{description}

%%% Local Variables: 
%%% mode: LaTeX
%%% TeX-master: "arbeit-hauptdatei"
%%% End: 

\chapter{Anwendung}
\label{cha:anwendung}

\section{Konzept der Umsetzung}
\label{sec:konz-der-umsetz}

\subsection{Seiteneffekte}
\label{sec:seiteneffekte}

\subsection{Einschr�nkung der Umsetzung}
\label{sec:einschr-der-umsetz}


\section{Schnittstellen nach au�en}
\label{sec:schn-nach-au3en}

\subsection{Schnittstelle zu A}
\label{sec:schnittstelle-zu}

\subsection{Schnittstelle zu B}
\label{sec:schnittstelle-zu-b}


%%% Local Variables: 
%%% mode: latex
%%% TeX-master: "arbeit-hauptdatei"
%%% End: 

\chapter{Ausblick}
\label{cha:ausblick}

\section{Zusammenfassung}
\label{sec:zusammenfassung}

\section{Weitere M�glichkeiten der Forschung}
\label{sec:weit-mogl-der}


%%% Local Variables: 
%%% mode: latex
%%% TeX-master: "arbeit-hauptdatei"
%%% End: 


\bibliographystyle{babalpha}
\bibliography{literatur}

\end{document}

%%% Local Variables: 
%%% mode: latex
%%% TeX-master: t
%%% End: 
