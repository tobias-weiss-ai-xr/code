\chapter{Theorie}
\label{cha:theorie}

Die Arbeit basiert im wesentlichen auf der Theorie des begrenzten
Wissens \cite{schlosser06:latexbuch}. Die Grundlagen des begrenzten
Wissens sowie diese Theorie wollen wir im Folgenden er�rtern.

\section{Grundlagen}
\label{sec:grundlagen}

Prinzipiell gilt, dass
\begin{equation}
  x=y+z
\end{equation}
unter der Annahme $x$ und $y$ als Zahlenma� von Textgr��e, $z$ als
Repr�sentation der Aufnahmef�higkeit.

\begin{equation*}
  \label{eq:pythagoras}
  \left.\begin{aligned}
  c^2&=a^2+b^2\\
  a^2&=p\cdot c\;\wedge\;
     b^2=q\cdot c\\
  h^2&=p\cdot q
  \end{aligned}\right\}
\begin{gathered}
  \text{Satzgruppe}\\
  \text{des Pythagoras}
\end{gathered}
\end{equation*}

\section{Theorie der B�ume}
\label{sec:theorie-der-baume}

Schon \cite{knuth:texbook} schreibt:

\begin{table}[htbp]
  \centering
  \caption{Vier Zahlen}
  \label{tab:vierzahlen}
  \begin{tabular}{lr}
    \toprule
    Zahl & Nummer \\
    \midrule
    Eins & Zwei \\
    Drei & Vier \\
    \bottomrule
  \end{tabular}
\end{table}

Mehr dazu finden Sie in Abschnitt~\ref{sec:seiteneffekte}.

\section{Erweiterungen der Theorie}
\label{sec:erwe-der-theor}

\begin{figure}[htbp]
  \centering
  %% Hier brauchen Sie eine existierende Grafikdatei
%  \includegraphics[width=.7\linewidth]{pinguin}
  \caption{Der Pinguin Tux}
  \label{fig:pinguin}
\end{figure}

\begin{figure}[htbp]
  \centering
  %% Eine Grafik in der Syntax von TikZ. Paket aktivieren!
%   \begin{tikzpicture}
%     \node {root}
%     child {node {left}}
%     child {node {right}
%       child {node {child}}
%       child {node {child}}
%     };
%   \end{tikzpicture}
  \caption{Ein Baum}
  \label{fig:einbaum}
\end{figure}

Das k�nnen wir machen durch
\begin{enumerate}
	\item etwas,
	\item etwas anders oder
	\item ganz etwas anderes.
\end{enumerate}

\begin{description}
	\item[Nichts] ist alles.
	\item[Alles] ist nichts.
\end{description}

%%% Local Variables: 
%%% mode: LaTeX
%%% TeX-master: "arbeit-hauptdatei"
%%% End: 
