%% Vorlage eines Konferenzbeitrags f�r das Buch
%% "Wissenschaftliche Arbeiten schreiben mit LaTeX"
%% von Joachim Schlosser, 2006-2009
%%
%% Sie d�rfen dieses Dokument frei verteilen, ver�ndern und verwenden.
%%
%% Dies ist nur eine Vorlage. 
%% Zwei Kommentarzeichen (%%) bedeuten, dass eine Erl�uterung folgt.
%% Ein Kommentarzeichen heisst, dass die Zeile deaktiviert wurde
%% Die meisten Pakete sind deaktiviert, Sie aktivieren Sie, indem Sie
%% das Kommentarzeichen am Beginn der Zeile entfernen.
%% Aktivieren Sie nur, was Sie wirklich brauchen!
%% Lesen Sie in l2tabu nach, ob nicht einige Pakete mittlerweile
%% veraltet sind, wenn Sie diese Vorlage erst 2010 oder sp�ter verwenden!

\documentclass[
%draft,                        %% um sp�ter den Zeilenumbruch zu kontrollieren
a4paper,                       %% DIN A4-Papier
BCOR4mm,                       %% Bindekorrektur 4 mm
DIVcalc,                       %% Satzspiegel berechnen
11pt,                          %% Schriftgr��e
tablecaptionabove,             %% Tabellen�ber"= statt unterschriften
twocolumn                      %% zweispaltig
]{scrartcl}
\usepackage[T1]{fontenc}       %% Schriftkodierung, die nativ Umlaute unterst�tzt
\usepackage[latin1]{inputenc}  %% Umlaute direkt eingeben
\usepackage{mathptmx}          %% schmal laufende Times als Brotschrift
\usepackage[scaled=.92]{helvet}%% Helvetica als Grotesk
% \usepackage[scaled=.87]{luximono} %% anderer Monotype-Font
\usepackage{geometry}          %% damit auch das Papierformat passt
\usepackage[english,german,ngerman]{babel}  %% Deutsch (neue Rechtschreibung) als Hauptsprache
\input{dehyphtex.tex}          %% zus�tzliche deutsche Trennmuster
%\usepackage{microtype}         %% Optischer Randausgleich mit pdfLaTeX
\usepackage{textcomp}          %% besondere Textzeichen
\usepackage{amssymb}           %% Mathesymbole
\usepackage{amsmath}           %% erweiterte Mathemakros
%\usepackage{paralist}          %% Aufz�hlungen im Flie�text
%\usepackage{mdwlist}           %% engere Aufz�hlungen und Nummerierungen
%\usepackage{filecontents}      %% eingebundene Dateien (f�r ltxtable)
%\usepackage{ltxtable}          %% lange Tabellen mit flexibler Spaltenbreite
%\usepackage{tabularx}          %% Tabellen mit flexibler Spaltenbreite 
\usepackage{booktabs}          %% sch�nere Tabellenlinien
\usepackage{graphicx}          %% Graphiken einbinden
\usepackage{tikz}              %% TikZ/pgf zum Schreiben von Grafiken
% \usepackage{ifpdf}            %% direktes Einbinden von Metapost-Graphiken
% \ifpdf 
% \DeclareGraphicsRule{*}{mps}{*}{} 
% \fi

%% Literaturverzeichnisse: Hier ist Babelbib gew�hlt. W�hlen Sie, was
%% Sie brauchen.
\usepackage{babelbib}          %% Mehrsprachiges Literaturverzeichnis

\usepackage[%
  bookmarks,                   %% PDF-Lesezeichen
  bookmarksopen=true,          %% Lesezeichenbaum aufgeklappt...
  bookmarksopenlevel=1,        %% ...um eine Ebene
  bookmarksnumbered=true,      %% Lesezeichen numerieren
%  pagebackref,                 %% Links von Literatur in Text
  pdfusetitle,                 %% LaTeX-Titelei als Metainfo nehmen
  pdfstartpage={1},            %% mit welcher Seite das PDF �ffnen
  pdfstartview={FitH},         %% Zoom auf Seitenbreite
  pdfkeywords={},              %% Stichw�rter f�rs PDF, kommagetrennt
  pdfsubject={},               %% Themenbeschreibung kurz
  pdfcreator={LaTeX with KOMA-Script and hyperref package},
%   pdfpagelabels,               %% Seitenzahlen wie im Dokument...
%   plainpages=false,            %% ...und nicht wie sonst
  hyperfootnotes=true,         %% Links auf Fu�noten
  hyperindex=true,             %% Indexeintr�ge verweisen auf Text
  linkbordercolor={0 1 1},     %% Rahmenfarbe interne Links
  menubordercolor={0 1 1},     %% Rahmenfarbe Literaturlinks
  urlbordercolor={1 0 0}       %% Rahmenfarbe externe Links
]{hyperref}                    %% Hyperlinks und Lesezeichen in PDF 
%\usepackage[figure]{hypcap}    %% Springt bei Links auf Abb. an die richtige Stelle

\pagestyle{headings}           %% lebende Kolumnentitel

\hyphenation{%
%  eventuelle Trennmusterdefinitionen, mit - als Trennstellenmarker  
}

%% Anpassungen f�r das Layout: Float-Pages
%\renewcommand{\floatpagefraction}{.7}  % vorher: .5

%% Gegen die meisten Overfull hboxes, aus
%% dctt, Axel Reichert, Message-ID: <a84us0$plqcm$7@ID-30533.news.dfncis.de>
\tolerance 1414
\hbadness 1414
\emergencystretch 1.5em
\hfuzz 0.3pt
\widowpenalty=10000
\vfuzz \hfuzz
\raggedbottom

%% Hier beginnt das eigentliche Dokument
\begin{document}

%% Die Eckdaten des Dokuments, wegen Babel erst nach \begin{document}
\author{Max Muster}
\title{Mein wissenschaftlicher Beitrag}
\date{24.\,12.~2009}

%% Titel daraus erzeugen
\maketitle

\begin{abstract}
Dieser Artikel basiert im wesentlichen auf der Theorie des begrenzten
Wissens \cite{schlosser06:latexbuch}. Die Grundlagen des begrenzten
Wissens sowie diese Theorie wollen wir im Folgenden er�rtern. 
\end{abstract}

\section{Theorie}
\label{cha:theorie}


\subsection{Grundlagen}
\label{sec:grundlagen}

Prinzipiell gilt, dass
\begin{equation}
  x=y+z
\end{equation}
unter der Annahme $x$ und $y$ als Zahlenma� von Textgr��e, $z$ als
Repr�sentation der Aufnahmef�higkeit.

\begin{equation*}
  \label{eq:pythagoras}
  \left.\begin{aligned}
  c^2&=a^2+b^2\\
  a^2&=p\cdot c\;\wedge\;
     b^2=q\cdot c\\
  h^2&=p\cdot q
  \end{aligned}\right\}
\begin{gathered}
  \text{Satzgruppe}\\
  \text{des Pythagoras}
\end{gathered}
\end{equation*}

\subsection{Theorie der B�ume}
\label{sec:theorie-der-baume}

Schon \cite{knuth:texbook} schreibt:

\begin{table}[htbp]
  \centering
  \caption{Vier Zahlen}
  \label{tab:vierzahlen}
  \begin{tabular}{lr}
    \toprule
    Zahl & Nummer \\
    \midrule
    Eins & Zwei \\
    Drei & Vier \\
    \bottomrule
  \end{tabular}
\end{table}

Mehr dazu finden Sie in Abschnitt~\ref{sec:konz-der-umsetz}.

\subsection{Erweiterungen der Theorie}
\label{sec:erwe-der-theor}

Nun ist es so, dass ausgenommen der nichtstandardisierten Verteilung
alle verteilten Standards nicht ausgenommen werden k�nnen.

Das k�nnen wir machen durch:
\begin{enumerate}
\item etwas,
\item etwas anders oder
\item ganz etwas anderes.
\end{enumerate}

\begin{figure}[htbp]
  \centering
  %% Eine Grafik in der Syntax von TikZ. Paket tikz deaktivieren wenn
  %% nicht ben�tigt!
  \begin{tikzpicture}
    \node {root}
    child {node {left}}
    child {node {right}
      child {node {child}}
      child {node {child}}
    };
  \end{tikzpicture}
  \caption{Ein Baum}
  \label{fig:einbaum}
\end{figure}

Diese Liste ist nat�rlich nicht als abschliessend zu betrachten und
kann beliebig erweitert werden. Etwa durch eine Beschreibungsliste:

\begin{description}
\item[Nichts] ist alles.
\item[Alles] ist nichts.
\end{description}

\section{Anwendung}
\label{cha:anwendung}

\subsection{Konzept der Umsetzung}
\label{sec:konz-der-umsetz}

Noch etwas tolles.

\subsection{Schnittstellen nach aussen}
\label{sec:schn-nach-au3en}

Noch etwas tolles.

\section{Schlussfolgerungen}
\label{sec:schlu3folgerungen}

Daraus k�nnen wir ein Resum�e ziehen: Ohne Inhalt keine Arbeit, wohl aber
einige Seiten Dokument.

\bibliographystyle{babalpha}
\bibliography{literatur}

\end{document}

%%% Local Variables: 
%%% mode: latex
%%% TeX-master: t
%%% End: 
