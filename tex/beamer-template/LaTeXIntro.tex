\documentclass{beamer}
\usepackage{amscd,amsfonts,amsmath,amssymb,latexsym,amsthm,geometry,booktabs,float}
\usepackage[mathscr]{eucal}
\usepackage{graphicx}
%\usepackage{movie15}

\newcommand\R{\mathbb{R}}
\newcommand\I{\mathbb{I}}
\newcommand\C{\mathcal{C}}
\newcommand\X{\mathcal{X}}
\newcommand\T{\mathcal{T}}
\newcommand\h{\mathcal{H}}
\newcommand\Oh{\mathcal{O}}
\newcommand\M{\mathcal{M}}
\newcommand\B{\mathcal{B}}

\mode<presentation>
\usetheme{Montpellier}%Berlin, classic, copenhagen, bars, Darmstadt, Dresden, Frankfurt, Ilmenau, lined, Malmoe, Montpellier, Singapore, Warsaw, Berkeley, Rochester
%\usetheme[secheader]{Madrid}
%\usecolortheme{whale}%seahorse, whale, dolphin, crane, beetle, rose, dove, fly, orchid, albatross, lily
%color setting more or less matching U of A colors

\setbeamercolor*{palette secondary}{use=structure,fg=white,bg=structure.fg!55!black}
\setbeamercolor*{palette tertiary}{use=structure,fg=white,bg=red!50!black}
\setbeamercolor*{palette quarternary}{use=structure,fg=white,bg=structure.fg!55!black}
\setbeamertemplate{navigation symbols}{}


\title[Intro to \LaTeX]{An Introduction to LaTeX and Beamer}
\subtitle{SWIG}
\author{Kevin LaTourette}
\institute[University of Arizona]{Program in Applied Mathematics, University of Arizona}
%\today

\AtBeginSection[]
{
  \begin{frame}<beamer>
    \frametitle{Outline}
    {
    \tableofcontents[currentsection]
        }
  \end{frame}
}

\begin{document}
\begin{frame}
  \titlepage
\end{frame}


\section{Introduction}
%%%%%%%%%%%%%%%%%%%%%%%%%%%%%%%%%%%
\begin{frame}
   \frametitle{\ldots Not like the gloves}
   \begin{itemize}
      \item $/'leit\epsilon x/$ \ldots ``Lay - tek'', ``Lah - tek'' is also commonly spoken
      
      \item A high quality typesetting (\emph{not a word processor}) language used by matematicains, scientists and scholars in academia and industry
      
      \item Generally regarded as the standard for writing academic papers
   \end{itemize}

\end{frame}

%%%%%%%%%%%%%%%%%%%%%%%%%%%%%%%%%%%
\subsection{Why use \LaTeX}
%%%%%%%%%%%%%%%%%%%%%%%%%%%%%%%%%%%
\begin{frame}
   \frametitle{Advantages of \LaTeX}
   
   \begin{itemize}
      \item Graduate students are poor.  \LaTeX is free!
      
      \item Available on all Windows, Mac \& Linux
      
      \item .tex files are ASCII files and very portable are portable, and the output is in PDF, DVI or PS.
      \begin{itemize}
         \item Every computer can open a PDF, not so with Word or Powerpoint
      \end{itemize}
   \end{itemize}
\end{frame}
%%%%%%%%%%%%%%%%%%%%%%%%%%%%%%%%
\begin{frame}
   \frametitle{Advantages continued\ldots}
   \begin{itemize}
      \item Typesetting is much better, more professional.  This is especially true for mathematics.  Can be formatted to suit various publication styles easily.
      
      \item Styles for different publication styles are easily available, as are Templates for letters, articles, books, reports etc. (\emph{just ask Google!})
      
      \item Bibliography management
   \end{itemize}

\end{frame}
%%%%%%%%%%%%%%%%%%%%%%%%%%%%%%%%%
\begin{frame}
   \frametitle{Still more advantages}
   \begin{itemize}
      \item Commands are easy to learn, though daunting at first
      
      \item Never crash!
      
      \item Can compile huge books\ldots $\approx70,000$ pages
   \end{itemize}

\end{frame}

%%%%%%%%%%%%%%%%%%%%%%%%%%%%%%%%
\begin{frame}
   \frametitle{Disadvantages of \LaTeX}
   \begin{enumerate}
      \item Font selection not as easy as Word.
      
      \item Difficult to flow text around pictures\slash figures
      
      \item If you write few documents, or only want short documents with little math
      \begin{enumerate}
        \item[a.)] The first time you write in \LaTeX, it will likely take longer than you would like\ldots
      \end{enumerate}
      
      \item Does not \emph{directly} support drawing figures\ldots you need other software
   \end{enumerate}
\end{frame}
%%%%%%%%%%%%%%%%%%%%%%%%%%%%%%%
\begin{frame}
   \frametitle{Disadvantages of \LaTeX, continued}
   \begin{enumerate}
      \item[5.] Spelling\slash Grammar not as convenient to check as in Word
      
      \item[6.] Must remember commands
      
      \item[7.] Not straightforward for creating complex tables
   \end{enumerate}
\end{frame}
%%%%%%%%%%%%%%%%%%%%%%%%%%%%%%%
\subsection{Where is \LaTeX}
\begin{frame}
   \frametitle{How do I get it?}
   \begin{itemize}
      \item \href{http://support.math.arizona.edu/tex/}{http://support.math.arizona.edu/tex/} for all your needs
      
      \item Good IDEs:
        \begin{enumerate}
           \item Kile (\emph{Linux})
           
           \item TeXnicCenter (\emph{Windows})
           
           \item WinEdt (\emph{Windows, clearly})
           
           \item TeXShop (\emph{Mac})
        \end{enumerate}
   \end{itemize}
\end{frame}
%%%%%%%%%%%%%%%%%%%%%%%%%%%%%%%%
\section{How do we use \LaTeX}
\subsection{Basic file}
\begin{frame}
   \frametitle{A Skeleton file}
      $\%$ my first LaTeX file!\\
      $\slash$ documentclass$[$12pt$]\{$article$\}$\\
        \hskip 1cm $\%$preamble\\
        \hskip 1cm $\slash$ usepackage$\{$graphix, amsmath$\}$\\
      $\slash$ begin$\{$document$\}$\\
         \hskip 1cm $\slash$ section$\{$This is a section title$\}$\\
         \hskip 1cm $\slash$ subsection$\{$This is a subsection title$\}$\\
         \hskip 1cm $\slash$ subsubsection$\{$This is\ldots you get the point$\}$\\
          \hskip 2cm $\slash$ par First Paragraph goes here.\\
          \vskip .5cm
          \hskip 2cm $\slash$ par Second Paragraph is here, and has some math.  $\$\slash$sqrt$\{ \slash$frac$\{\slash$alpha$\}\{\slash$pi$\}\} \slash$leq $\slash$epsilon$\$$\\
      \slash end$\{$document$\}$
\end{frame}
%%%%%%%%%%%%%%%%%%%%%%%%%%%%%%%%
\subsection{Some Examples}
\begin{frame}
   \frametitle{Real, live \LaTeX}
   \begin{itemize}
      \item For a comprehensive list of commands, go Google\ldots
   \end{itemize}
\end{frame}




\end{document}