% !TeX program = lualatex
% !TeX encoding = UTF-8 Unicode
\documentclass[ngerman,11pt,a4paper,paper=a4,oneside,parskip=half+]{scrarticle}
\usepackage{fontspec}
\usepackage{microtype}
%\setmainfont{LibertinusSans}
% ms arial compatible font
\setmainfont{Erewhon}
\usepackage[left=3cm, top=2.5cm, bottom=2.5cm, right=2.5cm% page margins
    %,showframe% <- only to show the page layout
]{geometry}
\usepackage{babel}
\usepackage[onehalfspacing]{setspace}
%page numbering right outer
\usepackage{scrlayer-scrpage}
\cfoot*{} %asterisk for first page overwrite
\rofoot*{\pagemark}
\setlength{\parindent}{0em} % no first line indent (Erstzeileneinzug)
%\setlength{\parskip}{1em} % gap between paragraphs (Zeilenzwischenraum)
\RedeclareSectionCommand[
    beforeskip=.3\baselineskip
]{paragraph}
% font and style concerns
\usepackage{mathtools}
\usepackage{amsmath,amssymb,amsfonts}
\usepackage{graphicx}
\usepackage{multirow}
\usepackage{array}
\usepackage{listings}
\usepackage{subcaption}
\usepackage{booktabs}
\usepackage{url}
\usepackage{natbib}
\setcitestyle{aysep={}} %no comma between author and year, e.g. (Scholl 1999, Kap. 1) for \citep citations
\renewcommand{\bibsection}{} % no bibliography title as section is used
\usepackage{lipsum} % dummy text
%algorithm visualisation
\usepackage{algorithm}
\usepackage{algpseudocode}
\floatname{algorithm}{Algorithmus}
%table caption on top
\usepackage{float}
\floatstyle{plaintop}
\restylefloat{table}
%full size tables
\usepackage{tabularx}
%rotate (e.g. table columns)
\usepackage{rotating}
%use , as decimal separator (german convention) in math mode
\usepackage{ziffer}
%adjust table size or float object positioning like pictures via adjustbox
\usepackage{adjustbox}

\begin{document}
\titlehead{MW31.6 -- Data and Knowledge Management}
\subject{Handout}
\title{Netzwerkdatenbankmodell}
%\subtitle{Untertitel}
\author{Tobias Weiß, Matr.-Nr: 159098\\tobias.weiss@uni-jena.de}
\date{\today}
%\publishers{Platz für Betreuer o.\,ä.}
\maketitle
\tableofcontents
% ----------------------------------------------------------------------------
\section{Motivation}
Dieser Abschnitt sollte sich mit der Aufgabenstellung befassen. Er kann auch
Grundlagen behandeln. Es kann jedoch sinnvoll sein, für die Grundlagen einen
eigenen Abschnitt zu verwenden.


\begin{figure}[htbp]
  \centering
  \includegraphics[width=0.5\linewidth]{example-image}
  \caption{Caption}
  \label{fig:figure}
\end{figure}

\fbox{
Test
}

\begin{equation}
    \varphi(v_k) = \frac{1}{1+e^{-v_k}} 
    \qquad
    \varphi'(v_k) = \varphi(v_k) \cdot (1 - \varphi(v_k))
\label{eq:sigmoid}
\end{equation}

\section{Durchführung}
Hier erzählt man nun, was man alles gemacht hat.
\section{Schluss}
Hierher gehört das Fazit und ggf. der Ausblick auf weitere Dinge, die getan
werden könnten.
\end{document}
%%% Local Variables:
%%% mode: latex
%%% TeX-engine: xetex
%%% TeX-PDF-mode: t
%%% coding: utf-8
%%% TeX-master: t
%%% End:  
